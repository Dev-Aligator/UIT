\documentclass[12pt]{article}
\usepackage[a4paper, left=3.17cm, right=3.17cm, top=2.54cm, bottom=2.54cm]{geometry}
\usepackage[T1]{fontenc}
\usepackage{mathptmx}
\usepackage{amsmath}
\usepackage{amsfonts}
\usepackage{chemformula}
\usepackage{cite}
\usepackage[colorlinks, linkcolor=black, anchorcolor=black, citecolor=black]{hyperref}
\usepackage{graphicx}
\usepackage[utf8]{vietnam}



\setlength{\parskip}{0.5em}
\title{Sự Khác Nhau Giữa Switch Và Router}
\author{\textup{9 - FOBE}}
\begin{document}
    \begin{titlepage}
\newcommand{\HRule}{\rule{\linewidth}{0.5mm}}
\includegraphics[width=8cm]{title/logo.png}\\[1cm] 
\center 
\quad\\[1.5cm]
\textsl{\Large The University of Information Technology}\\[0.5cm] 
\textsl{\large School of Information and Communication Technology (ICT)}\\[0.5cm] 
\makeatletter
\HRule \\[0.4cm]
{ \huge \bfseries \@title}\\[0.4cm] 
\HRule \\[1.5cm]
\begin{minipage}{0.4\textwidth}
\begin{flushleft} \large
\emph{Group:}\\
\@author 
\end{flushleft}
\end{minipage}
~
\begin{minipage}{0.4\textwidth}
\begin{flushright} \large
\emph{Supervisor:} \\
\textup{Tran Manh Hung}
\end{flushright}
\end{minipage}\\[3cm]
\makeatother
{\large An Assignment submitted for the Uit:}\\[0.5cm]
{\large \emph{IT005.N12.KHCL - Nhập môn mạng máy tính}}\\[0.5cm]
{\large \today}\\[2cm] 
\vfill 
\end{titlepage}

    \begin{abstract}
    Switch và Router là 2 thành phần quan trọng của một kết nối mạng. Mặc dù cả 2
    đều kết nối các thiết bị thông qua một mạng, chúng có những chức năng hoàn toàn
    khác nhau mà đôi khi người ta thường nhầm lẫn chúng là một. Bài này sẽ phân tích
    sự khác nhau giữa 2 thiết bị. \\
    
    \end{abstract}
    

    \section*{Switch}
    Switch là 1 thiết bị mạng cung cấp các chức năng chia sẻ thông tin và tài nguyên
    bằng cách kết nối những thiết bị mạng như \textbf{máy tính, máy in và servers} trong một
    kết nối mạng nhỏ. \
    
    Thông qua Switch, các thiết bị được kết nối có thể chia sẻ dữ liệu, thông tin và
    giao tiếp với nhau. Nếu không có Switch chúng ta không thể xây dựng các hệ thống
    mạng nhỏ cũng như không thể kết nối các thiết bị trong phạm vi tòa nhà hoặc khu vực. \

    Hiện nay có 2 loại Switch chính được sử dụng là: Unmanaged và Managed Switch.
    
    
    \subsection*{Cách hoạt động}
    Như đã biết, mỗi kết nối mạng chứa một \textbf{MAC (Media Access Control) riêng biệt.}
    Khi một thiết bị hay máy tính gửi một gói IP đến một thiết bị khác, sau đó switch đặt
    gói IP trên với địa chỉ MAC nguồn và đích, đóng gói lại thành 1 Frame sau đó gửi
    đến thiết bị khác. Khi Frame đến thiết bị đích, nó được giải nén và thiết bị nhận được gói IP.

    \subsection*{Ưu điểm}
    \begin{itemize}
        \item Nâng cao băng thông của hệ thống mạng.
        \item Có thể kết nối trực tiếp đến workstations hoặc thiết bị.
        \item Cải thiện hiệu năng của kết nối mạng.
        \item Giảm tải công việc trên máy chủ.
    \end{itemize}
    
    
    \section*{Router}
    Router là một thiết bị mạng được sử dụng để kết nối nhiều switch và hệ thống
    mạng tương ứng của chúng để xây dựng một hệ thống mạng lớn hơn. Các switch này
    và mạng của chúng có thể ở cùng 1 vị trí hoặc ở những vị trí khác nhau. \

    Router là một thiết bị thông minh chịu trách nhiệm định tuyến các gói thông tin
    từ nguồn đến đích thông qua một mạng. \textbf{Nó còn phân phát hoặc định tuyến kết nối mạng
    từ modem đến mọi thiết bị mạng có dây hoặc không dây như Pc, Laptop, Mobile Phone, tablet, ... }\

    Router có 2 chức năng chính
    \begin{itemize}
        \item Tạo và duy trì \textbf{mạng LAN}
        \item Quản lí dữ liệu vào ra.
    \end{itemize}

    Router kết nối nhiều mạng và cho phép các thiết bị được kết nối truy cập internet. Nó hoạt động
    trên \textbf{network layer} và định tuyến các gói dữ liệu thông qua con đường
    ngắn nhất trong mạng.\\

    Hiện có 2 loại router chính được sử dụng là: \textbf{Wireles and Wired/Broadband Router}

    \subsection*{Cách hoạt động}
    Trong nhà hay văn phòng, chúng ta có nhiều thiết bị kết nối mạng và với router, các thiết
    bị này có thể kết nối internet và hình thành một mạng. \

    Router định tuyến/truyền các gói dữ liệu với địa chỉ IP đã biết từ mạng này 
    đến mạng khác hoặc trong cùng một mạng. \

    Nó tìm ra con đường truyền tốt và nhanh nhất và gửi dữ liệu từ đó đến các thiết
    bị kết nối trong phạm vi mạng. \
    
    \subsection*{Ưu điểm}
    \begin{itemize}
        \item Kết nối được với nhiều cấu trúc mạng khác nhau.
        \item Bảo mật tốt với mật khẩu.
        \item Giảm ùn tắc mạng.
        \item Cung cấp các gói thông tin chính xác với bộ định tuyến thông minh.
    \end{itemize}

    \section*{Sự khác nhau giữa Switch và Router}
    \begin{flushleft}
    Chức năng chính của switch là để kết nối những thiết bị cuối trong
    khi router dùng để kết nối 2 mạng khác nhau. 

    Switch nhắm tới việc xác định địa chỉ đến của gói IP nhận được và
    chuyển tiếp nó đến địa chỉ đó. Trong khi đó nhiệm vụ chính của router là
    tìm đường truyền ngắn và tốt nhất cho các gói dữ liệu dựa trên bảng định tuyến.

    Có rất nhiều kĩ thuật switching như circuit switching, packet switching, 
    and message switching. Để so sánh router chỉ sử dụng 2 kĩ thuật là adaptive routing
    and non-adaptive routing.

    Switch lưu trữ MAC address trong lookup table or bảng CAM để lấy địa chỉ
    nguồn và đích. Trái ngược router lưu trữ các địa chỉ IP trong bảng định tuyến

    Switch hoạt động dựa trên data link layer, router là network layer. 
    \end{flushleft}
    
    \section*{Bảng so sánh}
    \begin{center}
        \begin{tabular}{ p{0.35\linewidth} | p{0.6\linewidth}}
        \hline
        Switch & Router  \\ \hline
        Kết nối nhiều thiết bị mạng trong cùng một mạng. & Kết nối nhiều switch và hệ thống 
        mạng tương ứng. \\ \hline
        Hoạt động trên data link layer. & Hoạt động trên network layer. \\ \hline
        Được sử dụng ở phạm vi LAN. & Có thể sử dụng ở cả LAN hay MAN. \\ \hline
        Không thể thực hiện NAT & Có thể thực hiện NAT \\ \hline
        Cần nhiều thời gian khi đưa ra lựa chọn định tuyến phức tạp. & Thực hiện nhanh hơn switch gấp nhiều lần. \\ \hline
        Chỉ cung cấp port security. & Cung cấp những phương án để bảo vệ mạng khỏi những 
        mối đe dọa. \\ \hline
        Chỉ là một thiết bị semi-Intelligent. & Được biết đến là một thiết bị Intelligent. \\
        \hline
        Chế độ truyền dữ liệu bán song công hoặc song công toàn phần. & Chỉ song công toàn phần nhưng ta hoàn toàn có thể thay đổi một cách 
        thủ công để hoạt đông bán song công. \\ \hline
        Gửi thông tin từ thiết bị này sang thiết bị khác theo dạng Frame hoặc theo
        các packet. & Gửi thông tin từ mạng này sang mạng khác theo dạng data packet. \\ \hline
        Chỉ có thể hoạt động với mạng có dây & Cả mạng có dây và không dây \\ \hline
        Khả dụng ở nhiều port & Mặc định chỉ có 2 port là Fast Ethernet Port nhưng ta có thể thêm các serial port. \\ \hline
        Sử dụng bảng CAM cho địa chỉ MAC nguồn và đích. & Sử dụng bảng định tuyến để tìm ra tuyến tốt nhất cho IP.
        \end{tabular}
    \end{center}

    \section*{Conclusion}
    Chúng ta có thể kết luận rằng cả hai đều là những thiết bị quan trọng
    để xây dựng một hệ thống mạng và chúng có những ưu điểm riêng trong mạng. Tuy nhiên để
    xây dựng một home-based network và kết nối các thiết bị ta cần Switch, và để kết nối 2 mạng với nhau
    ta dùng Router.
    \bibliographystyle{ieeetrans}

\end{document}