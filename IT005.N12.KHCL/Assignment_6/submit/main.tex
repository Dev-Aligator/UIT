\documentclass[12pt,a4paper]{article}
\usepackage{amsmath,amsthm,amsfonts,amssymb,amscd}
\usepackage{times}     
\usepackage{bookmark}         
\usepackage{graphicx}           
\usepackage{float}              
\usepackage{booktabs}           
\usepackage{xcolor}             
\usepackage{geometry}           
\usepackage{fullpage}           
\usepackage{comment}                     
\usepackage{listings}           
\usepackage{lastpage}           
\usepackage{fancyhdr}           
\usepackage{hyperref}           
\usepackage[small,bf]{caption}  
\usepackage{multicol}
\usepackage{tikz}               
\usepackage{circuitikz}         
\usepackage{verbatim}          
\usepackage{cite}               
\usepackage[us]{datetime} 
\usepackage{blindtext}
\usepackage[utf8]{vietnam}
\usepackage{array}
\usepackage{makecell}
\usepackage{tabularx}
\usepackage{titlesec}
\usepackage{enumitem}
\usepackage{multirow}
\setlength\parindent{0pt}

%%%%%%%%%%%%%%%%%%%%%%%%%%%%%%%%%%%%%%%%%%%%%%%%%%%%%%%%%%%%%%
\titleformat{\section}
{\color{UM_DarkBlue}\normalfont\large\bfseries}
{\color{UM_DarkBlue}\thesection}{1em}{}

%%%%%%%%%%%%%%%%%%%%%%%%%%%%%%%%%%%%%%%%%%%%%%%%%%%%%%%%%%%%%%
\hypersetup{
    draft=false,
    final=true,
    colorlinks=true,
    citecolor=UM_DarkBlue,
    anchorcolor=yellow,
    linkcolor=UM_DarkBlue,
    urlcolor=UM_DarkBlue,
    filecolor=green,      
    pdfpagemode=FullScreen,
    bookmarksopen=false
    }
    
%%%%%%%%%%%%%%%%%%%%%%%%%%%%%%%%%%%%%%%%%%%%%%%%%%%%%%%%%%%%%%
\lstdefinestyle{Fortran}{
basicstyle=\scriptsize,        % the size of the fonts that are used for the code
  breakatwhitespace=false,         % sets if automatic breaks should only happen at whitespace
  breaklines=false,                 % sets automatic line breaking
  captionpos=b,                    % sets the caption-position to bottom
  commentstyle=\color{mygreen},    % comment style
  extendedchars=true,              % lets you use non-ASCII characters; for 8-bits encodings only, does not work with UTF-8
  keepspaces=true,                 % keeps spaces in text, useful for keeping indentation of code (possibly needs columns=flexible)
  keywordstyle=\color{blue},       % keyword style
  language=[95]Fortran,                 % the language of the code
  numbers=left,                    % where to put the line-numbers; possible values are (none, left, right)
  numbersep=5pt,                   % how far the line-numbers are from the code
  numberstyle=\tiny\color{mygray}, % the style that is used for the line-numbers
  rulecolor=\color{black},         % if not set, the frame-color may be changed on line-breaks within not-black text (e.g. comments (green here))
  showspaces=false,                % show spaces everywhere adding particular underscores; it overrides 'showstringspaces'
  showstringspaces=false,          % underline spaces within strings only
  showtabs=false,                  % show tabs within strings adding particular underscores
  stepnumber=1,                    % the step between two line-numbers. If it's 1, each line will be numbered
  stringstyle=\color{mymauve},     % string literal style
  tabsize=4,                       % sets default tabsize to 2 spaces
  title=\lstname                   % show the filename of files
}

%%%%%%%%%%%%%%%%%%%%%%%%%%%%%%%%%%%%%%%%%%%%%%%%%%%%%%%%%%%%%%%
\definecolor{UM_Brown}{HTML}{3D190D}
\definecolor{UM_DarkBlue}{HTML}{2264B0}
\definecolor{UM_LightBlue}{HTML}{1CA9E1}
\definecolor{UM_Orange}{HTML}{fEB415}




%\newcommand{\tu}[1]{\textup{#1}}
\newcommand{\tu}[1]{\mathrm{#1}}







\newcommand{\ones}{\mathbf 1}
\newcommand{\reals}{{\mbox{\bf R}}}
\newcommand{\integers}{{\mbox{\bf Z}}}
\newcommand{\symm}{{\mbox{\bf S}}}  % symmetric matrices

\newcommand{\nullspace}{{\mathcal N}}
\newcommand{\range}{{\mathcal R}}
\newcommand{\Rank}{\mathop{\bf Rank}}
\newcommand{\Tr}{\mathop{\bf Tr}}
\newcommand{\diag}{\mathop{\bf diag}}
\newcommand{\card}{\mathop{\bf card}}
\newcommand{\rank}{\mathop{\bf rank}}
\newcommand{\conv}{\mathop{\bf conv}}
\newcommand{\prox}{\mathbf{prox}}

\newcommand{\Expect}{\mathop{\bf E{}}}
\newcommand{\Prob}{\mathop{\bf Prob}}
\newcommand{\Co}{{\mathop {\bf Co}}} % convex hull
\newcommand{\dist}{\mathop{\bf dist{}}}
\newcommand{\argmin}{\mathop{\rm argmin}}
\newcommand{\argmax}{\mathop{\rm argmax}}
\newcommand{\epi}{\mathop{\bf epi}} % epigraph
\newcommand{\Vol}{\mathop{\bf vol}}
\newcommand{\dom}{\mathop{\bf dom}} % domain
\newcommand{\intr}{\mathop{\bf int}}
\newcommand{\sign}{\mathop{\bf sign}}

\newcommand{\cf}{{\it cf.}}
\newcommand{\eg}{{\it e.g.}}
\newcommand{\ie}{{\it i.e.}}
\newcommand{\etc}{{\it etc.}}

\graphicspath{{Figs/}}

\begin{document}

\textcolor{UM_Brown}{
\begin{minipage}{0.1\textwidth}
    \begin{flushleft}
        \includegraphics[height=3.5cm]{logo.eps}
    \end{flushleft}
\end{minipage}
\begin{minipage}{0.8\textwidth}
    \begin{center}
        \textbf{\Large Nhập môn mạng máy tính}\\
        \vspace{5pt}
        Assignment 6 \\
        \vspace{10pt}
        \textit{Group 9 - FOBE} \\
        \vspace{5pt}
        \longdate\today
    \end{center}
\end{minipage}
\vspace{10pt}
\hrule
}

\section*{Problem 4:}
\subsection*{A.}
\begin{table}[H]
    \centering
    \begin{tabular}{|c|c|}
    \toprule
     \bf{Địa chỉ đích}  & \textbf{Link interface} \\
     \midrule
     H3 & \#3 \\
     \bottomrule
    \end{tabular}
\end{table}

\subsection*{B.}
Không thể có bảng forwarding đề yêu cầu. Vì đối với mạng datagram, quy tắc chuyển tiếp chỉ dựa trên địa chỉ đích.

\subsection*{C.}
\begin{table}[H]
    \caption*{Router A}
    \centering
    \begin{tabular}{|c|c|c|c|}
    \toprule
     \bf{Cổng vào}  & \textbf{Số hiệu của kết nối ảo vào} & \textbf{Cổng ra} & \textbf{Số hiệu của kết nối ảo ra} \\
     \midrule
     1 & 12 & 3 & 22 \\
     \midrule
     2 & 63 & 4 & 18 \\
     \bottomrule
    \end{tabular}
\end{table}

\subsection*{D.}
\begin{table}[H]
    \centering
    \begin{tabular}{|c|c|c|c|c|}
    \toprule
     \bf{Router} & \bf{Cổng vào}  & \textbf{Số hiệu của kết nối ảo vào} & \textbf{Cổng ra} & \textbf{Số hiệu của kết nối ảo ra} \\
     \midrule
     B & 1 & 22 & 2 & 24 \\
     \midrule
     C & 1 & 18 & 2 & 50 \\
     \midrule
     \multirow{2}{*}{D} &  1 & 24 & 3 & 70 \\
                        &  2 & 50 & 3 & 76 \\
     \bottomrule
    \end{tabular}
\end{table}

\newpage
\section*{Problem 5}
\subsection*{A.}
Since the VC number is 2-bit, the maximum number of links that could be established is 22 = 4 links. Therefore, there is no VC number that could be assigned to the new VC.
\subsection*{B.}
The VC number is 2-bit and the number of links is 4. \\
The number of combinations could be used: 24 = 16.

\section*{Problem 10}
\subsection*{A.}
\begin{table}[H]
    \centering
    \begin{tabular}{|c|c|}
    \toprule
     \bf{Prefix Match} & \bf{Link Interface}  \\
     \midrule
     11100000\;00  & 0 \\
     \midrule
     11100000\;01000000 & 1 \\
     \midrule
     11100000 & 2 \\
     \midrule
     11100001\;0 &  2 \\
     \midrule
     otherwise & 3 \\
     \bottomrule
    \end{tabular}
\end{table}
\subsection*{B.}
\begin{itemize}
    \item Prefix trùng ứng với địa chỉ thứ 1 là entry số  5: link interface 3.
    \item Prefix trùng ứng với địa chỉ thứ 2 là entry số  2: link interface 1.
    \item Prefix trùng ứng với địa chỉ thứ 3 là entry số  5: link interface 3.
\end{itemize}

\section*{Problem 11}
Range of destination host addresses:
\begin{equation*}
    00 = 00-000000 \rightarrow 00-111111 
\end{equation*}
\begin{equation*}
    010 = 010-00000 \rightarrow 01-011111
\end{equation*}
\begin{equation*}
    011 = 011-00000 \rightarrow 011-11111
\end{equation*}
\begin{equation*}
    10 = 10-000000 \rightarrow 10-111111
\end{equation*}
\begin{equation*}
    11 = 11-000000 \rightarrow 11-111111
\end{equation*}
Number of addresses in the range:
\begin{itemize}
    \item Number of addresses for interface 0 = 2$^6$ = 64.
    \item Number of addresses for interface 1 = 2$^5$ = 32.
    \item Number of addresses for interface 2 = 2$^5$ +2$^6$ = 32 + 64 = 96.
    \item Number of addresses for interface 3 = 2$^6$ = 64.
\end{itemize}

\section*{Problem 12}
Range of destination host addresses:
\begin{equation*}
    1 = 10-000000 \rightarrow 10-111111 
\end{equation*}
\begin{equation*}
    10 = 110-00000 \rightarrow 110-11111
\end{equation*}
\begin{equation*}
    111 = 111-00000 \rightarrow 111-11111
\end{equation*}
\begin{equation*}
    Otherwise = 0-0000000 \rightarrow 0-1111111
\end{equation*}
Number of addresses in the range:
\begin{itemize}
    \item Number of addresses for interface 0 = 2$^6$ = 64.
    \item Number of addresses for interface 1 = 2$^5$ = 32.
    \item Number of addresses for interface 2 = 2$^5$ = 32.
    \item Number of addresses for interface 3 = 2$^7$ = 128.
\end{itemize}

\section*{Contributors}
\begin{table}[H]
    \centering
    \begin{tabular}{c|c}
    \toprule
     \bf{Problem }  & \textbf{Contributors} \\
     \midrule
     4 & Trâm Anh \\
     5 & Kim Yến \\
     10 & Hoàng Tân \\
     11, 12 & Gia Khang \\
     \bottomrule
    \end{tabular}
\end{table}



\end{document} 